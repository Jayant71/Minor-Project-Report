\chapter{Implementation}
\vspace{-1.5cm}
\hspace{-1cm}\rule{19cm}{0.4pt} 

\section{Development Environment}
\subsection{Software Tools and Frameworks}
\begin{itemize}
    \item \textbf{Programming Languages:} Python was chosen as the primary programming language due to its extensive support for scientific computing, machine learning, and data manipulation. Libraries such as NumPy, Pandas, and SciPy were integral for data processing, while Matplotlib and Seaborn were utilized for data visualization.
    
    \item \textbf{Machine Learning Libraries:} PyTorch~\cite{Paszke2019PyTorch} was employed for building and executing machine learning models, particularly for experimentation tasks involving neural networks. These libraries provide robust frameworks for developing complex models efficiently.
    
    \item \textbf{Natural Language Processing Tools:} For text generation and language model tasks, Hugging Face's Transformers library was leveraged. This integration facilitates handling text-based experiments and automating aspects of paper writing.
    
    \item \textbf{Version Control:} Git was used for version control to ensure that all code changes were properly tracked. GitHub served as a cloud repository to store and share code along with proper documentation, enabling collaboration among team members.
    
    \item \textbf{Integrated Development Environment (IDE):} Jupyter Notebook was utilized for prototyping and testing smaller portions of code, while Visual Studio Code was used for developing and managing the overall project structure.
\end{itemize}

\subsection{Hardware Resources}
\begin{itemize}
    \item \textbf{Computational Power:} The project relied on cloud-based platforms equipped with high-performance GPUs to run experiments, particularly those involving large-scale machine learning tasks. Services like Google Cloud Platform (GCP) and AWS EC2 instances were utilized.
    
    \item \textbf{Storage Solutions:} Data, experiment results, and model checkpoints were stored using cloud storage solutions such as AWS S3 and Google Cloud Storage. This approach provides easy access and scalability for managing large datasets.
\end{itemize}

\subsection{Collaboration Tools}
\begin{itemize}
    \item \textbf{Project Management:} Tools like Trello or Jira were implemented to organize tasks, assign responsibilities, and track progress throughout the project lifecycle.
    
    \item \textbf{Communication:} Slack was employed for real-time communication among team members to facilitate discussions regarding issues and updates.
\end{itemize}

\section{Project Implementation}
\subsection{Execution Stages}
\begin{enumerate}[leftmargin=2cm, labelwidth=1.5cm]
    \item[\textbf{Stage 1:}] \textbf{Idea Generation and Experiment Design} \\
    The first stage was the generation of research ideas based on predefined templates and input parameters. The ideas were then screened for novelty using APIs such as Semantic Scholar, which cross-referenced existing research to ensure that the proposed concepts were novel~\cite{Shea2024PlaneSearch}. Based on these validated ideas, experiment designs were developed, with an emphasis on feasibility and computational efficiency.

    \item[\textbf{Stage 2:}] \textbf{Experimentation and Data Collection} \\
    The automated experiments were run using a combination of available datasets and generated data. To account for variability and get robustness, the system ran multiple iterations of each experiment. All experiment parameters, results, and configurations were logged and stored for later analysis.

    \item[\textbf{Stage 3:}] \textbf{Analysis and Result Documentation} \\
    Analysis was conducted on the collected data through statistical tools after performing the experiments. Key metrics were used to evaluate accuracy, novelty score, and computational efficiency for each experiment. The findings were recorded in an academic format; the results were summarized visually, and further details could be found in the report narrative.

    \item[\textbf{Stage 4:}] \textbf{Peer Review and Evaluation} \\
    The last step was an automated review to evaluate the quality of the produced research. The system reviewed its own papers with an internal quality control mechanism to ensure that all parts of the paper were written according to academic standards. This was followed by an external review from simulated peers to validate the results further.
\end{enumerate}

\section{Project Timeline}
\begin{enumerate}[leftmargin=2cm, labelwidth=1.5cm]
    \item[\textbf{Phase 1:}] \textbf{Planning and Setup (Weeks 1-2)} \\ % chktex 8
    Defining Project Scope and Objectives: Establishing clear goals and outlining the project's aims.
    Setting Up Development Environment and Tools: Configuring software tools, frameworks, and hardware resources necessary for the project.
    Preparing Initial Datasets and Templates: Compiling relevant datasets and creating templates for experimentation to ensure readiness for subsequent phases.

    \item[\textbf{Phase 2:}] \textbf{Idea Generation and Experiment Design (Weeks 3-4)} \\ % chktex 8
    Generating and Validating Research Ideas: Utilizing predefined templates and input parameters to create innovative research ideas, followed by novelty assessment using APIs.
    Designing Experiments: Developing detailed experimental designs based on validated ideas, focusing on feasibility and computational efficiency.
    Implementing the System for Autonomous Experiments: Setting up the automated system to conduct experiments without manual intervention.

    \item[\textbf{Phase 3:}] \textbf{Experimentation and Data Collection (Weeks 5-6)} \\ % chktex 8
    Executing Experiments: Running automated experiments using existing datasets and generated data to collect results.
    Implementing Feedback Loops: Refining experimental designs based on initial results to enhance robustness and reliability.

    \item[\textbf{Phase 4:}] \textbf{Data Analysis and Documentation (Weeks 7-8)} \\ % chktex 8
    Analyzing Experimental Data: Applying statistical tools and machine learning methods to identify trends and insights from the collected data.
    Generating Visualizations and Writing Reports: Creating visual representations of the data and documenting findings in an academic format.

    \item[\textbf{Phase 5:}] \textbf{Review and Final Evaluation (Weeks 9-10)} \\ % chktex 8
    Conducting Internal and External Reviews: Evaluating the quality of research outputs through an automated review process followed by simulated peer reviews.
    Refining Final Documentation: Making necessary adjustments to the report based on feedback received during the review stage, preparing for submission.
\end{enumerate}

\section{Resource Management}
\subsection{Human Resources}
\begin{itemize}
    \item \textbf{Project Manager:} Oversaw overall progress, ensured timely delivery and effective communication
    \item \textbf{Lead Developer:} Responsible for core automation system coding and experiment design
    \item \textbf{Data Scientist:} Handled data collection, analysis, and visualization
    \item \textbf{Research Specialist:} Generated research ideas, validated experiments, and wrote documentation
\end{itemize}

\subsection{Computational Resources}
\begin{itemize}
    \item Cloud platforms (GCP) utilized 70\% of computational budget
    \item Local machines with high-performance GPUs used for development and testing
\end{itemize}

\subsection{Financial Resources}
\begin{itemize}
    \item 60\% - Computation expenses, cloud storage, and GPU costs % chktex 8
    \item 40\% - Tool subscriptions, APIs, development tools, and project management software % chktex 8
\end{itemize}


\section{Challenges Faced}
\begin{itemize}
    \item \textbf{Computational Constraints:} Running large-scale experiments, particularly those involving complex machine learning models, proved resource-intensive. The need to adjust parameters so that they could be contained within the available budget and time constraints often hindered the efficiency of the experimentation process. This challenge called for careful planning and prioritization to ensure that experiments critical to the overall plan could be run without overstepping resource limits.
    
    \item \textbf{Error Handling in Automated Systems:} The automation pipeline was error-prone in terms of code execution at times or experiment runs showing some erratic behavior. Debugging such errors was very time-consuming and required constant adjustments to the system's logic. This was an important lesson to learn to ensure proper error handling mechanisms and rigorous testing protocols are in place to minimize the disruptions in the research process.
    
    \item \textbf{Limitations of Novelty Detection:} The novelty-checking process was not flawless, failing sometimes to identify redundant research ideas. This made it necessary to continue improving the novelty detection algorithms in order to make them more accurate and reliable. Overcoming this challenge called for constant assessment and adjustment of methods applied to evaluate idea uniqueness.
\end{itemize}

\section{Lessons Learned}
\begin{itemize}
    \item \textbf{Flexibility:} Flexibility in adapting to unexpected challenges, such as hardware limitations or unanticipated errors in the system, proved to be crucial in maintaining progress.

    \item \textbf{Continuous Improvement:} Each stage of the system design required constant iteration as opportunities to refine the process continued to emerge.

    \item \textbf{Data Quality Matters:} Quality, diverse data sets are critical for the creation of valid and reproducible results.

    \item \textbf{Ethical Oversight:} All automatic systems designed and implemented into research should be based upon ethical considerations, especially against bias, transparency, and data privacy.
\end{itemize}
