\chapter{Methodology}
\vspace{-1.5cm}
\hspace{-1cm}\rule{19cm}{0.4pt} 

\section{Project Overview}
This project aims to create an automated system that can generate, test, and document scientific ideas with minimal human intervention on repetitive research tasks, without compromising rigor and quality. The methodology covers the whole pipeline of research, which has been divided into five major stages: idea generation, experimental design, data collection, analysis, and documentation.\\
It first creates the potential research ideas with input parameters or a predefined starting template, then it evaluates those ideas as to whether they are new and feasible using external sources such as academic databases, after which it designs the experiments for testing the remaining concepts. It does this through the development of a complete experimental setup that provides all the details on the method by which data will be gathered and analyzed.

\vspace{1cm}

\begin{figure}[h]
    \centering
    \begin{tikzpicture}[node distance=2cm]
        % Nodes
        \node (userInput) [entity] {User Input};
        \node (process) [process, right of=userInput, xshift=8cm] {Automated Research Workflow};
        \node (output) [data, below of=process, yshift=-3cm, xshift=-4cm] {Research Paper};
        
        % Arrows
        \draw [arrow] (userInput) -- (process) node[midway, above] {Template + Arguments};
        \draw [arrow] (process) -- (output) node[midway, right] {Final Document}; 
    \end{tikzpicture}
    \caption{ High-level overview of the research generation workflow.}
    \label{fig:level1} % chktex 24
\end{figure}

The experiments are therefore designed, conducted, and their data collected in an organized and systematic manner. At this level, the most important goal is to achieve reliable as well as relevant results against the hypotheses proposed. Consequently, the obtained data is analysed by using relevant statistical means and the outcomes are therefore presented in a structured reporting format that abides by conventional standards of professional academic work. This documentation includes interpretations of the findings, discussions on their implications, and suggestions for future research. To measure the success of the project, a combination of quantitative metrics, such as novelty scores and reproducibility rates, and qualitative reviews will be used. This includes simulated peer reviews, providing an understanding of how well the automated system produces valuable research outputs and whether it complies with scientific standards. \\
The project will be initially tested within a computational domain to ensure practicality and effectiveness. Following this phase, the system's adaptability for broader fields of research will be evaluated, assessing its potential application in various scientific disciplines beyond its original scope.

\begin{figure}[htbp]
    \centering
    \begin{tikzpicture}[node distance=2cm]
        % Nodes
        \node (template) [entity] {Template Folder};
        \node (cmd) [entity, below of=template, yshift=-1cm] {User Commands};
        \node (ideaGen) [process, right of=template, xshift=4cm] {Idea Generation};
        \node (noveltyCheck) [process, right of=ideaGen, xshift=4cm] {Novelty Checking};
        \node (experiments) [process, below of=noveltyCheck, yshift=-2cm] {Perform Experiment};
        \node (logs) [data, below of=experiments, yshift=-3cm] {Logs + Notes};
        \node (writeup) [process, left of=logs, xshift=-4cm] {Paper Writeup};
        \node (output) [data, left of=writeup, xshift=-4cm] {PDF Output};
         
        % Arrows 
        \draw [arrow] (template) -- (ideaGen) node[midway, above] {Template};
        \draw [arrow] (cmd) -- (ideaGen) node[midway, right] {Arguments};
        \draw [arrow] (ideaGen) -- (noveltyCheck) node[midway, above] {Seed Ideas};
        \draw [arrow] (noveltyCheck) -- (experiments) node[midway, left] {Novel Ideas};
        \draw [arrow] (experiments) -- (logs) node[midway, left] {Experiment Results};
        \draw [arrow] (logs) -- (writeup) node[midway, above] {Notes};
        \draw [arrow] (writeup) -- (output) node[midway, above] {LaTeX PDF};
    \end{tikzpicture}
    \caption{Main modules of the automated research generation system.}
    \label{fig:level2} % chktex 24
\end{figure}


\begin{figure}[ht]
    \centering
    \begin{tikzpicture}[node distance=2cm]
        % Nodes
        \node (start) [entity] {User Starts};
        \node (seed) [process, right of=start, xshift=4cm] {Seed Ideas};
        \node (genIdeas) [process, right of=seed, xshift=4cm] {Generate New Ideas};
        \node (novelty) [process, below of=genIdeas, yshift=-3cm] {Check Novelty};
        \node (experiments) [process, left of=novelty, xshift=-4cm, ] {Run Experiments};
        \node (logs) [data, left of=experiments, xshift=-4cm] {Generate Logs};
        \node (writeup) [process, below of=logs, yshift=-2cm] {Generate Writeup};
        \node (output) [data, right of=writeup, xshift=4cm] {Final PDF};
        \node (review) [process, right of=output, xshift=3cm] {Review Papers};
        \node (genreview) [data, below of=review, yshift=-4cm] {Reviews Generated};
 
        % Arrows 
        \draw [arrow] (start) -- (seed);
        \draw [arrow] (seed) -- (genIdeas) node[midway, above] {Ideas};
        \draw [arrow] (genIdeas) -- (novelty) node[midway, right] {New Ideas};
        \draw [arrow] (novelty) -- (experiments) node[midway, above] {Novel Ideas};
        \draw [arrow] (experiments) -- (logs) node[midway, above] {Results};
        \draw [arrow] (logs) -- (writeup) node[midway, right] {Notes};
        \draw [arrow] (writeup) -- (output) node[midway, above] {Draft Paper};
        \draw [arrow] (output) -- (review) node[midway, above] {Paper};
        \draw [arrow] (review) -- (genreview) ;
    \end{tikzpicture}
    \caption{ Detailed system processes for automated research generation.}
    \label{fig:level3} % chktex 24  
\end{figure}


\section{Research Design and Approach}
The research design for this project is set as an iterative, modular process with feedback loops to enable continuous improvement. The dominant approach follows a number of key stages that are important to the overall functionality of the automated system.\\
The first stage focuses on idea generation through two main components. Through computational models, the system creates a diverse set of hypotheses or research questions. This is done through randomization and guided generation based on existing knowledge within a scope. Tools such as semantic search APIs are integrated to assess the novelty of ideas generated, ensuring that the ideas are novel and do not duplicate work already done.\\
The experimental design stage involves converting hypotheses into testable forms. Every formed hypothesis is turned into a testable hypothesis and then used for designing experiments according to specific domain needs. Code template libraries are modified so the system can dynamically change experiment designs in response to user feedback from preliminary testing rounds. This improves versatility and reactivity.\\
For data gathering, the system implements experiments by conducting them on preselected sets or simulated scenarios to generate measurable data. Multiples of the experiments are run using automation to establish robustness and reliability of the results.\\
The analysis and documentation phase involves two key components. Results are analyzed through statistical tools or machine learning pre-programmed to help determine the patterns and insights that come out. The findings are then presented in an academic manner using visualizations and structured text according to professional research reporting standards.\\
Finally, the evaluation stage incorporates an automated review mechanism that evaluates the clarity, originality, and potential impact of the documented work, simulating a peer-review process. This provides valuable feedback on the quality of the research outputs.


\section{Data Collection Methods}
Data collection in this project is fully automated, utilizing a combination of pre-existing datasets, simulated environments, and self-generated experimental results. The process implements multiple sophisticated methods to ensure comprehensive and reliable data gathering across various research scenarios.\\
The foundation of our data collection strategy lies in the integration of predefined datasets. The system carefully identifies and incorporates datasets that are most relevant to the research domain being investigated. These datasets may come from public repositories, academic databases, or be generated internally by the system depending on the specific requirements of the project. This approach ensures that the research begins with a solid foundation of reliable data, which can be used for both initial analysis and as a benchmark for comparing new results.\\
Simulated experiments form another crucial component of our data collection methodology. In scenarios where real-world data collection is impractical, costly, or simply impossible, the system employs sophisticated simulation environments. These simulated environments are carefully designed to mirror real-world conditions while offering the flexibility to manipulate variables in ways that might not be possible in actual settings. This approach allows for extensive experimentation without the limitations and constraints typically associated with real-world data collection, enabling the exploration of various scenarios and hypotheses in a controlled environment.\\
The system implements comprehensive real-time data logging throughout all experimental processes. Every aspect of the experiments, including intermediate results, error messages, performance metrics, and system states, is meticulously recorded. These detailed logs serve multiple purposes: they provide a complete audit trail of the experimental process, enable thorough analysis of the results, and facilitate the reproduction of experiments when needed. The logging system is designed to capture both successful outcomes and failures, as both types of results can provide valuable insights for the research process.\\
Dynamic data collection capabilities represent the most advanced aspect of our methodology. The system is engineered to adaptively modify experimental parameters during runtime, responding to intermediate results and changing conditions. This dynamic approach allows for the collection of data across a broad spectrum of experimental conditions, ensuring that the research captures a comprehensive view of the phenomenon being studied. The system can automatically adjust variables, measurement frequencies, and other parameters based on preliminary results, optimizing the data collection process for maximum insight and efficiency.

\section{Project Directory}
\subsection{Structure}
\begin{lstlisting}
.env
example_papers/
ai_scientist/
    generate_ideas.py
    llm.py
    perform_experiments.py
    perform_writeup.py
    perform_review.py
data/
launch_scientist.py
results/
templates/
LICENSE
README.md
requirements.txt
\end{lstlisting}

\subsection{Explanation of Structure}

\begin{description}
  \item[\texttt{.env}] Environment variables file, used to configure environment-specific settings (API keys).
  \item[\texttt{example\_papers/}] Directory containing example research papers generated by the system.
  \item[\texttt{ai\_scientist/}] Directory containing the main modules of the AI Scientist application.
  \item[\texttt{generate\_ideas.py}] Script to generate research ideas.
  \item[\texttt{llm.py}] Script for fetching data from the Language Model API. % chktex 13
  \item[\texttt{launch\_scientist.py}] Script to launch the AI Scientist application.
  \item[\texttt{perform\_experiments.py}] Script to perform experiments.
  \item[\texttt{perform\_writeup.py}] Script to perform write-up of results.
  \item[\texttt{templates/}] Directory containing templates for various documents.
  \item[\texttt{README.md}] Readme file providing an overview of the project.
  \item[\texttt{results/}] Directory to store results of experiments.
  \item[\texttt{LICENSE}] License file for the project.
  \item[\texttt{requirements.txt}] List of Python dependencies required for the project.
\end{description}

\section{Data Analysis Techniques}
This process of data analysis for the project aims at deriving meaningful insights with a high validity and reproducibility of the results. The most basic statistical methods such as mean, standard deviation, and confidence intervals are used to summarize and assess the outcomes of experiments. It is basically an easy foundational analysis of understanding central tendencies and variability.
Comparison of the results from different experimental setups is used to assess the efficiency of various approaches. Comparative analysis includes evaluation against the baseline results or established benchmarks, which can provide an understanding of how different methodologies perform relative to one another.
Data visualizations are crucial in the analytical process. The system develops plots, graphs, and heatmaps to depict trends, relationships, and anomalies in data. Libraries like Matplotlib or Seaborn are often used to create them in Python and enhance the interpretability of the results.\\
Error and failure analysis is also performed when experiments do not deliver expected results. In such cases, potential flaws in design or execution can be pinpointed. It's a crucial feedback loop in ensuring that the research process stays adaptive and responsive for the improvements of the following iterations.\\
Finally, to avoid losing clarity and academic value in the presentation of findings, the system uses natural language generation techniques for summarization. This ensures findings are communicated effectively, but they are also accessible to a wider audience while holding scholarly standards. Overall, this comprehensive data analysis process supports the goal of producing high-quality research outputs by the project.

\section{Ethical Considerations}
This project recognizes the ethical concerns of automating scientific research and follows principles intended to mitigate risks and encourage responsible use. One of the key concerns is transparency: all outputs generated by automation, such as research results and reviews, are appropriately labeled as system-generated in order to maintain accountability. Transparency is essential to preserve trust in the research process.\\Another key emphasis is on the mitigation of bias. Proactive efforts in the form of mitigation are made regarding biases generated while coming up with ideas, while selecting the data and when interpreting results by using varied datasets and strict evaluation criteria to enhance objectivity from research outputs.\\Another area of concern is related to data privacy. When one uses external datasets, the project ensures compliance with data protection laws and ethical guidelines when not collecting or analyzing sensitive data or personal data. So, in such a regard, data privacy protects and respects individual rights and complies with the ethical and moral code in research practice. Safeguards are provided for the system to not create unethical or harmful research ideas. A review mechanism exists that flags potentially problematic outputs so that timely intervention and correction can be undertaken.\\Responsible deployment is a guiding principle of this project. The system is designed to assist and complement human researchers, not replace them. The outputs are designed to inspire and guide further exploration, not definitive conclusions. Through the emphasis on collaboration between automated systems and human expertise, the project promotes a responsible and ethical approach to scientific inquiry.

\section{Limitations}
While the project demonstrates significant advancements in automating research workflows, it is not without limitations. These challenges must be acknowledged to ensure a realistic understanding of the system's capabilities and areas for improvement.
\begin{enumerate}
    \item \textbf{Domain Dependency}\\
    The effectiveness of the system is highly dependent on the availability of structured datasets and domain-specific knowledge. Certain fields may require additional customization to ensure that the outputs generated are meaningful and relevant. This dependency can limit the system's applicability across diverse research areas, necessitating tailored approaches for different domains.

    \item \textbf{Computational Constraints}\\
    Running experiments, particularly those involving large datasets or complex simulations, can be resource-intensive and costly. The computational demands may pose challenges for users with limited access to high-performance computing resources, potentially restricting the system's widespread adoption.

    \item \textbf{Error Propagation}\\
    Errors that occur in one stage of the research pipeline—such as experiment design—can propagate to subsequent stages, potentially compromising the final output. This risk highlights the importance of rigorous validation and quality control measures throughout the entire process to minimize the impact of errors.

    \item \textbf{Limited Context Understanding}\\
    Although the system is designed to generate results and reports, it lacks the nuanced understanding that a human researcher possesses. This limitation may result in overly simplistic interpretations of complex findings, underscoring the need for human oversight in interpreting results and drawing conclusions.

    \item \textbf{Ethical and Safety Concerns}\\
    There is an inherent risk of misuse, such as generating low-quality or misleading research outputs. Ensuring oversight and regulation is critical to mitigate these risks and uphold ethical standards in research. Continuous monitoring and evaluation mechanisms will be necessary to address potential ethical concerns associated with automated research generation.
\end{enumerate} 